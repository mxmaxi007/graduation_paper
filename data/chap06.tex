\chapter{总结与展望}
\label{cha:summary_prospect}

\section{本文工作总结}
\label{sec:summary}

随着人机交互日应用的日益普遍,语音情感识别正在逐渐受到更多研究者的的重视。本文主要对语音情感识别中特征抽取和选择的方式进行了研究,包括基于传统声学特征的语音情感识别框架和基于深度学习的端到端的语音情感识别框架。针对现有研究中存在的关于特征选取,特征抽取和变长语音段输入相关的问题,提出了一些有效的解决方案。本文主要的工作可以分为以下几个部分:

\textbf{一、提出一种基于情感对的语音情感识别框架,为不同的情感对选择不同的声学特征,并在最后的决策融合过程中引入心理学的情感空间模型,从而提升了系统的识别率。} 传统的语音情感识别系统通常为所有的情感选取相同的声学特征来完成最后的情感分类,但实验证明不同的情感和不同的声学特征的相关性并不同。针对这一问题,我们将分别为不同的情感对选取不同的特征集合,将原先的多分类问题转变为多个二分类问题,并在最后的决策融合过程中通过贝叶斯分类器引入情感空间的信息。在公开的情感语音数据集IEMOCAP上,我们方法取得了比传统的识别框架更好的准确率。

\textbf{二、构建了基于深度神经网络的端到端的语音情感识别系统,使用语谱图代替传统的声学特征,从而提升了系统的识别率。} 随着深度学习技术和工具的发展,许多的研究者开始采用深度神经网络在原始语音信号上直接构建分类或者回归模型,被称之为端到端的系统。相比于采用传统的声学特征,这种方法可以抽取到更符合任务目标的特征表示。语谱图是语音信号的一种无损表示,我们通过卷积神经网络来从语谱图上直接抽取和情感相关的特征表示,然后通过循环神经网络来建模语音信号的时序信息,最后通过全连接网络将输出映射到不同情感的后验概率。相比于采用传统的声学特征,端到端的系统能够取得更好的准确率。

\textbf{三、设计了一种能够处理变长语音段的神经网络结构来实现端到端的系统,消除了语音分段带来的中性语音和情感语音的混淆,从而提升了系统的识别率。} 在使用深度神经网络实现端到端的语音情感识别系统时,由于卷积神经网络和循环神经网络很难处理变长的输入,通常会把变长的语音句子切分成多段等长的语音段,然后将所有语音段都标记为对应句子的情感标签,但这样会导致部分中性语音段被标记为有情感。针对于这一问题,我们采用补齐和掩码的方式来处理神经网络中变长的输入序列,避免了错误标注带来了的效果变差。相对于切分等长语音段的方法,我们直接输入整个变长语音句子的方法可以在相同的数据集上取得更好的准确率。

\section{未来工作展望}
\label{sec:prospect}

我们的工作已经探索一些关于情感相关的特征抽取和选择的问题,包括传统的机器学习模型和深度学习模型,但在现有工作的基础上仍然有许多问题可以继续进行研究。

在提出的基于情感对的语音情感识别框架中,我们只采用比较小的声学特征集合来进行特征选择,有很多的声学特征都没有考虑进来。如果将声学特征集合扩大,本文采用的特征选择算法会需要较长的时间来执行,所以可以进一步采用更为快速的特征选择算法,例如遗传算法等。此外,本文只采用一些比较简单的分类模型来预测情感,但仍然有许多其他的分类模型没有被测试过,因此系统的识别率仍然有很大的提升空间。

在我们提出的基于深度学习的端到端语音情感识别系统中,也只是使用了CNN、RNN等几种深度神经网络结构。现在已经有许多新的神经网络结构被提出,并且在其他的一些序列分类问题上取得了很不错的效果,例如注意力的机制,这些新的神经网络结构或许可以进一步提升系统的识别率。此外,情感语音数据的获取比较困难,所以训练数据比较少,而深度学习模型通常都有比较多的参数,需要大量的训练数据来学习参数,只使用少量数据会导致模型训练时过拟合,所以我们需要设计一些无监督的方法来增加训练数据,例如先用少量的标记数据去训练一个简单模型,然后去筛选未标记的数据。除了增加数据以外,也可以通过经验加入一些规则,从而能够较少模型的学习负担。