\thusetup{
  %******************************
  % 注意:
  %   1. 配置里面不要出现空行
  %   2. 不需要的配置信息可以删除
  %******************************
  %
  %=====
  % 秘级
  %=====
  secretlevel={秘密},
  secretyear={10},
  %
  %=========
  % 中文信息
  %=========
  ctitle={语音情感识别中特征选择的研究},
  cdegree={工程硕士},
  cdepartment={计算机科学与技术系},
  cmajor={计算机技术},
  cauthor={马习},
  csupervisor={吴志勇\hspace*{24pt}副研究员},
  % cassosupervisor={陈文光教授}, % 副指导老师
  % ccosupervisor={某某某教授}, % 联合指导老师
  % 日期自动使用当前时间,若需指定按如下方式修改:
  % cdate={超新星纪元},
  %
  % 博士后专有部分
  cfirstdiscipline={计算机科学与技术},
  cseconddiscipline={系统结构},
  postdoctordate={2009年7月——2011年7月},
  id={编号}, % 可以留空: id={},
  udc={UDC}, % 可以留空
  catalognumber={分类号}, % 可以留空
  %
  %=========
  % 英文信息
  %=========
  etitle={Feature Selection in Speech Emotion Recognition},
  % 这块比较复杂,需要分情况讨论:
  % 1. 学术型硕士
  %    edegree:必须为Master of Arts或Master of Science(注意大小写)
  %             “哲学、文学、历史学、法学、教育学、艺术学门类,公共管理学科
  %              填写Master of Arts,其它填写Master of Science”
  %    emajor:“获得一级学科授权的学科填写一级学科名称,其它填写二级学科名称”
  % 2. 专业型硕士
  %    edegree:“填写专业学位英文名称全称”
  %    emajor:“工程硕士填写工程领域,其它专业学位不填写此项”
  % 3. 学术型博士
  %    edegree:Doctor of Philosophy(注意大小写)
  %    emajor:“获得一级学科授权的学科填写一级学科名称,其它填写二级学科名称”
  % 4. 专业型博士
  %    edegree:“填写专业学位英文名称全称”
  %    emajor:不填写此项
  edegree={Master of Engineering},
  emajor={Computer Technology},
  eauthor={Ma Xi},
  esupervisor={Associate Professor Wu Zhiyong},
  % eassosupervisor={Chen Wenguang},
  % 日期自动生成,若需指定按如下方式修改:
  % edate={December, 2005}
  %
  % 关键词用“英文逗号”分割
  % ckeywords={\TeX, \LaTeX, CJK, 模板, 论文},
  % ekeywords={\TeX, \LaTeX, CJK, template, thesis}
}

% 定义中英文摘要和关键字
\begin{cabstract}
  在人机交互系统逐渐变得普遍的今天,只是理解语音中的语言学信息已经不足以满足所有需求,何提取语音中的情感信息在许多的应用场景中也变得越来越重要。传统的语音情感识别可以分为情感相的关声学特征的提取和情感分类模型的构建两部分,原始语音通常会先被映射到情感信息相关的声学特征,然后采用各种分类模型将将特征向量映射到对应的情感类别。近年来,随着深度学习方法的发展与普及,深度神经网络开始越来越多被应用到这一领域,并且取得了不错的效果。此外,特征提取和情感分类两个部分也开始被整合到一起,通过深度神经网络将可以构建从原始语音直接到情感类别的端到端的识别系统。但是如何为不同情感选择特定的相关特征以及处理长度变化的语音,并没有被现有的研究广泛的关注。针对这两个问题,本文将分别从传统的语音情感识别方法和端到端的深度学习方法入手,设计对应的方法来提升系统的识别率。主要的研究工作和和贡献如下:
  
  % 引入情感相关的心理学信息以及处理长度变化的语音,并没有被现有的研究广泛的关注。
  
 
  \textbf{一、提出一种基于情感对的语音情感识别框架,为不同的情感对选择不同的声学特征,并在最后的决策融合过程中引入心理学的情感空间模型,从而提升了系统的识别率。} 传统的语音情感识别系统通常为所有的情感选取相同的声学特征来完成最后的情感分类,但实验证明不同的情感和不同的声学特征的相关性并不同。针对这一问题,我们将分别为不同的情感对选取不同的特征集合,将原先的多分类问题转变为多个二分类问题,并在最后的决策融合过程中通过贝叶斯分类器引入情感空间的信息。在公开的情感语音数据集IEMOCAP上,我们方法取得了比传统的识别框架更好的准确率。

  \textbf{二、设计了一种能够处理变长语音段的神经网络结构来实现端到端的系统,消除了语音分段带来的中性语音和情感语音的混淆,从而提升了系统的识别率。} 在使用深度神经网络实现端到端的语音情感识别系统时,由于卷积神经网络和循环神经网络很难处理变长的输入,通常会把变长的语音句子切分成多段等长的语音段,然后将所有语音段都标记为对应句子的情感标签,但这样会导致部分中性语音段被标记为有情感。针对于这一问题,我们采用补齐和掩码的方式来处理神经网络中变长的输入序列,避免了错误标注带来了的效果变差。相对于切分等长语音段的方法,我们直接输入整个变长语音的方法可以在相同的数据集上取得更好的准确率。

  % \begin{itemize}
  %   \item \textbf{提出一种基于情感对的语音情感识别框架,并在最后的决策融合过程中引入心理学的情感空间模型,从而提升了系统的识别率。} 传统的语音情感识别系统通常为所有的情感选取相同的声学特征来完成最后的情感分类,但实验证明不同的情感和不同的声学特征的相关性并不同。针对这一问题,我们将分别为不同的情感对选取不同的特征集合,将原先的多分类问题转变为多个二分类问题,并在最后的决策融合过程中通过贝叶斯分类器引入情感空间的信息。在公开的情感语音数据集IEMOCAP上,我们方法取得了比传统的识别框架更好的准确率。
  %   \item \textbf{设计了一种能够处理变长语音段的神经网络结构,消除了语音分段带来的中性语音和情感语音的混淆,从而提升了系统的识别率。} 在使用深度神经网络实现端到端的语音情感识别系统时,由于卷积神经网络和循环神经网络很难处理变长的输入,通常会把变长的语音句子切分成多段等长的语音段,然后将所有语音段都标记为对应句子的情感标签,但这样会导致部分中性语音段被标记为有情感。针对于这一问题,我们采用补齐和掩码的方式来处理神经网络中变长的输入序列,避免了错误标注带来了的效果变差。相对于切分等长语音段的方法,我们直接输入整个变长语音的方法可以在相同的数据集上取得更好的准确率。
  % \end{itemize}

  % \begin{enumerate}
  %   \item 用例子来解释模板的使用方法;
  %   \item 用废话来填充无关紧要的部分;
  %   \item 一边学习摸索一边编写新代码。
  % \end{enumerate}
 
\end{cabstract}

% 如果习惯关键字跟在摘要文字后面,可以用直接命令来设置,如下:
\ckeywords{语音情感识别;情感对;情感空间模型;变长语音段;深度神经网络}

\begin{eabstract}
   An abstract of a dissertation is a summary and extraction of research work
   and contributions. Included in an abstract should be description of research
   topic and research objective, brief introduction to methodology and research
   process, and summarization of conclusion and contributions of the
   research. An abstract should be characterized by independence and clarity and
   carry identical information with the dissertation. It should be such that the
   general idea and major contributions of the dissertation are conveyed without
   reading the dissertation.

   An abstract should be concise and to the point. It is a misunderstanding to
   make an abstract an outline of the dissertation and words ``the first
   chapter'', ``the second chapter'' and the like should be avoided in the
   abstract.

   Key words are terms used in a dissertation for indexing, reflecting core
   information of the dissertation. An abstract may contain a maximum of 5 key
   words, with semi-colons used in between to separate one another.
\end{eabstract}

\ekeywords{\TeX, \LaTeX, CJK, template, thesis}
