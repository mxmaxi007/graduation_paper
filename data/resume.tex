\begin{resume}

  \resumeitem{个人简历}

  1992 年 6 月 28 日出生于 甘肃 省 天水 市。

  2010 年 9 月考入 北京邮电 大学 计算机 系 网络工程 专业,2014 年 7 月本科毕业并获得 工学 学士学位。

  2015 年 8 月考入 清华 大学 计算机科学与技术 系攻读 计算机工程硕士 学位至今。

  \researchitem{发表的学术论文} % 发表的和录用的合在一起

  % 1. 已经刊载的学术论文(本人是第一作者,或者导师为第一作者本人是第二作者)
  \begin{publications}
    \item \textbf{X. Ma}, Z.Y. Wu, J. Jia, M.X. Xu, H. Meng and L.H. Cai “Speech Emotion Recognition with Emotion-Pair based Framework Considering Emotion Distribution Information in Dimensional Emotion Space,” Proc. Interspeech, 2017. (CCF C类,EI收录,检索号:20175204591394,Oral Paper)
    \item \textbf{X. Ma}, D. Wang, J. Tejedor “Similar Word Model for Unfrequent Word Enhancement in Speech Recognition,” IEEE/ACM Transactions on Audio, Speech, and Language Processing, vol 24, no. 10, 2016. (CCF B类,SCI 收录,检索号:WOS:000381442600012,Regular Paper)
    \item \textbf{X. Ma}, X. Wang, D. Wang and Z. Zhang, “Recognize foreign low-frequency words with similar pairs”, Proc. Interspeech, pp. 458-462, 2015. (CCF C类,EI 收录,检索号:20160902029708,Oral Paper)
    \item \textbf{X. Ma}, X. Wang and D. Wang, “Low-frequency word enhancement with similar pairs in speech recognition”, Proc. IEEE China Summit \& International Conference on Signal and Information Processing, pp. 343-347, 2015. (EI 收录,检索号:20160701912086,Oral Paper)
  \end{publications}

  \researchitem{参加的科研项目} % 发表的和录用的合在一起
  \begin{achievements}
    \item 国家自然科学基金项目:面向自然口语对话的深层次信息感知与表达方法研究(资助号:61375027)
    \item 国家自然科学基金重点项目:互联网话语理解的心理机制与计算建模(资助号:61433018)
    \item 国家社会科学基金重大项目:社会情感的语音生成与认知的跨语言跨文化研究(资助号:13\&ZD189)
  \end{achievements}

  % % 2. 尚未刊载,但已经接到正式录用函的学术论文(本人为第一作者,或者
  % %    导师为第一作者本人是第二作者)。
  % \begin{publications}[before=\publicationskip,after=\publicationskip]
  %   \item Yang Y, Ren T L, Zhu Y P, et al. PMUTs for handwriting recognition. In
  %     press. (已被 Integrated Ferroelectrics 录用. SCI 源刊.)
  % \end{publications}

  % % 3. 其他学术论文。可列出除上述两种情况以外的其他学术论文,但必须是
  % %    已经刊载或者收到正式录用函的论文。
  % \begin{publications}
  %   \item Wu X M, Yang Y, Cai J, et al. Measurements of ferroelectric MEMS
  %     microphones. Integrated Ferroelectrics, 2005, 69:417-429. (SCI 收录, 检索号
  %     :896KM)
  %   \item 贾泽, 杨轶, 陈兢, 等. 用于压电和电容微麦克风的体硅腐蚀相关研究. 压电与声
  %     光, 2006, 28(1):117-119. (EI 收录, 检索号:06129773469)
  %   \item 伍晓明, 杨轶, 张宁欣, 等. 基于MEMS技术的集成铁电硅微麦克风. 中国集成电路,
  %     2003, 53:59-61.
  % \end{publications}
\end{resume}
